%%%%%%%%%%%%%%%%%%%%%%%%%%%%%%%%%%%%%%%%%
% Friggeri Resume/CV
% XeLaTeX Template
% Version 1.0 (5/5/13)
%
% This template has been downloaded from:
% http://www.LaTeXTemplates.com
%
% Original author:
% Adrien Friggeri (adrien@friggeri.net)
% https://github.com/afriggeri/CV
%
% License:
% CC BY-NC-SA 3.0 (http://creativecommons.org/licenses/by-nc-sa/3.0/)
%
% Important notes:
% This template needs to be compiled with XeLaTeX and the bibliography, if used,
% needs to be compiled with biber rather than bibtex.
%
%%%%%%%%%%%%%%%%%%%%%%%%%%%%%%%%%%%%%%%%%

\documentclass{friggeri-cv} % Add 'print' as an option into the square bracket to remove colors from this template for printing
\usepackage[pdfauthor={Charlie Meyer},
            pdftitle={Charlie Meyer résumé},
            pdfsubject={Charlie Meyer résumé},
            pdfkeywords={Charlie Meyer résumé},
            pdfproducer={XeLateX},
            pdfcreator={Xelatex}]{hyperref}
\usepackage{lastpage}
\usepackage{fancyhdr}
\fancyhf{}
%\cfoot{\makebox[\textwidth][c]{page \thepage~of~\pageref{LastPage}}}
\addbibresource{bibliography.bib} % Specify the bibliography file to include publications
\pagestyle{fancy}
\begin{document}

\header{charlie}{meyer}{software artist, computer scientist, open source enthusiast, tinkerer} % Your name and current job title/field

%----------------------------------------------------------------------------------------
%	SIDEBAR SECTION
%----------------------------------------------------------------------------------------

\begin{aside} % In the aside, each new line forces a line break
\section{contact}
mailing address available upon request
Austin, Texas
~
630.886.7082
~
\href{mailto:charlie@charliemeyer.net}{charlie@charliemeyer.net}
\href{https://charliemeyer.net}{https://charliemeyer.net}
\href{https://www.github.com/cemeyer2}{Github}
\href{www.linkedin.com/in/cemeyer2}{LinkedIn}
~
%\emph{Please do not contact me if you work for a recruiting agency.}
\section{tongue}
native english,
professional spanish proficiency
\end{aside}

\section{objective}

\textbf{To obtain} a full time position in the software engineering field at a dynamic high tech company.

%----------------------------------------------------------------------------------------
%	WORK EXPERIENCE SECTION
%----------------------------------------------------------------------------------------
%\pagebreak
\section{industry experience}

%------------------------------------------------
\begin{entrylist}
\entry
{2014-Present}
{HomeAway}
{Austin, Texas}
{\emph{Senior Software Engineer} \\
At HomeAway, I am a the lead software engineer on our internal tools and infrastructure team. I was previously on the core platform team which covers HomeAway's high-traffic REST API and enterprise messaging busses. I daily deal with technologies such as RabbitMQ, Elasticsearch, Hadoop, Kafka, and Docker among many others. \\
Detailed achievements:
\begin{itemize}
\item Architected, developed, and rolled out a new internal application based on Ruby on Rails used by all engineers to accurately capture capitalizable time
\item Architected and led a team to develop a distributed SOA-based JSON registry and dashboard framework
\item Mentored the 3 Day Startup Program sponsored by HomeAway both in Austin, TX and London, UK
\item Architected, developed, and open-sourced the \href{https://github.com/homeaway/homeaway_api_ruby}{HomeAway Ruby SDK for the HomeAway Developer API}
\item Open-sourced a \href{https://github.com/homeaway/homeaway-storm}{Ruby library for communicating with Apache Storm clusters} on behalf of HomeAway
%\item Led and architected the quality effort to roll out a new caching infrastructure for our high traffic API based on Elasticsearch and Cassandra
%\item Provided continual platform support to engineers and quality assurance team members across the organization
\end{itemize}}
\end{entrylist}
\begin{entrylist}
\entry
{2012--2014}
{IBM}
{Austin, Texas}
{\emph{Staff Software Engineer} \\
Responsible for Power7+ AIX bring-up including integrated system test as well as an integral member of the
Cluster Aware AIX team, the backbone of the clustering capabilities of the IBM enterprise grade UNIX operating system,
which is an underlying component of IBM PowerHA and VIOS SSP. As part of that team, I led the effort to maintain, refactor,
and enhance the AIX cluster communications daemon. \\
Detailed achievements:
\begin{itemize}
\item Led the AIX effort to certify Oracle Real Application Clusters database on PowerHA
\item Enhanced AIX clustering to scale to 32 nodes and 1024 disks per node, including parallelization of core clustering libraries
\item Enabled unicast AIX cluster heartbeating
\item Enabled dynamic network configuration change support across clustered systems
\item Co-author of a clustering algorithm patent 
\end{itemize}}
\end{entrylist}
%------------------------------------------------
%\begin{entrylist}
%\entry
%{2011-2012}
%{IBM}
%{Champaign, Illinois}
%{\emph{Campus Representative} \\
%Campus representative for the University of Illinois at Urbana-Champaign. Coordinated with IBM HR and recruiters to
%organize events to promote IBM on campus to students across campus. Actively pursued and recommended new candidates for engineering positions.}
%\end{entrylist}
%------------------------------------------------
\begin{entrylist}
\entry
{2010-2011}
{IBM}
{Austin, Texas}
{\emph{Power Systems Test Engineer Co-op} \\
Architected and developed an automated functional test suite composed of approximately 130 test cases for IBM Power Systems
firmware using Rational tools, including Rational Functional Tester and Rational ClearQuest. Deployed existing and new code bases to Rational Quality Manager for usage by remote teams. Performed intensive Power Systems debugging, installs, and triage.}
\end{entrylist}
%------------------------------------------------
\begin{entrylist}
\entry
{2008-2009}
{IBM}
{Rochester, Minnesota}
{\emph{Power Systems Performance Co-op} \\
Worked as a member of a team to benchmark systems, analyze performance data, develop workloads, perform
maintenance on IBM Power Systems, and develop tools to aid in performance analysis. 
\begin{itemize}
\item Created an automation framework to streamline TPC-C benchmarks
\item Developed the front-end and CL/C++ harness for a new implementation of TPC-E for IBM i
\item Architected and developed an API to access Collection Services, PEX, JobWatcher, and DiskWatcher data from IBM i in Java
\end{itemize}}
\end{entrylist}
%------------------------------------------------
\begin{entrylist}
\entry
{2007-2008}
{Solidware Technologies}
{Champaign, Illinois}
{\emph{Software Engineer Intern} \\
Worked as a member of an agile team developing a J2EE web based application which performed static code analysis.} \\
%Key responsibilities included:
%\begin{itemize}
%\item Developing new application functionality
%\item End to end QA, including unit testing, integration testing, and regression testing
%\item Writing product documentation and transforming it across a variety of mediums
%\item Participating in product launch strategy
%\item Translating business needs into technical requirements
%\end{itemize}}
\end{entrylist}

%----------------------------------------------------------------------------------------
%	EDUCATION SECTION
%----------------------------------------------------------------------------------------

\section{education}

\begin{entrylist}
%------------------------------------------------
\entry
{2010-2012}
{Masters of Science {\normalfont computer science}}
{University of Illinois, Urbana-Champaign}
{Specialization: Software Engineering, Software Architecture, Design Patterns, Next Generation Computer Science Education \\ 
Thesis: CoMoTo: The Collaboration Modeling Toolkit }
%------------------------------------------------
\entry
{2005-2010}
{Bachelor of Science {\normalfont computer science}}
{University of Illinois, Urbana-Champaign}
{Specialization: Software Engineering, Information Assurance}
%------------------------------------------------
\end{entrylist}
%------------------------------------------------

%----------------------------------------------------------------------------------------
%	PUBLICATIONS SECTION
%----------------------------------------------------------------------------------------

\section{publications}

\printbibsection{patent}{patents}

\printbibsection{thesis}{thesis}

\printbibsection{inproceedings}{international peer-reviewed conferences/proceedings} % Print all miscellaneous entries from the bibliography

\printbibsection{report}{technical reports} % Print all research reports from the bibliography

%----------------------------------------------------------------------------------------

\end{document}
