%%%%%%%%%%%%%%%%%%%%%%%%%%%%%%%%%%%%%%%%%
% Friggeri Resume/CV
% XeLaTeX Template
% Version 1.0 (5/5/13)
%
% This template has been downloaded from:
% http://www.LaTeXTemplates.com
%
% Original author:
% Adrien Friggeri (adrien@friggeri.net)
% https://github.com/afriggeri/CV
%
% License:
% CC BY-NC-SA 3.0 (http://creativecommons.org/licenses/by-nc-sa/3.0/)
%
% Important notes:
% This template needs to be compiled with XeLaTeX and the bibliography, if used,
% needs to be compiled with biber rather than bibtex.
%
%%%%%%%%%%%%%%%%%%%%%%%%%%%%%%%%%%%%%%%%%

\documentclass{friggeri-cv} % Add 'print' as an option into the square bracket to remove colors from this template for printing
\usepackage[pdfauthor={Charlie Meyer},
            pdftitle={Charlie Meyer résumé},
            pdfsubject={Charlie Meyer résumé},
            pdfkeywords={Charlie Meyer résumé},
            pdfproducer={XeLateX},
            pdfcreator={Xelatex}]{hyperref}
\usepackage{lastpage}
\usepackage{fancyhdr}
\fancyhf{}
%\cfoot{\makebox[\textwidth][c]{page \thepage~of~\pageref{LastPage}}}
\addbibresource{bibliography.bib} % Specify the bibliography file to include publications
\pagestyle{fancy}
\begin{document}

\header{charlie}{meyer}{software artist, computer scientist, open source enthusiast} % Your name and current job title/field

%----------------------------------------------------------------------------------------
%	SIDEBAR SECTION
%----------------------------------------------------------------------------------------

\begin{aside} % In the aside, each new line forces a line break
\section{contact}
\href{mailto:charlie@charliemeyer.net}{charlie@charliemeyer.net}
\href{https://www.github.com/cemeyer2}{Github}
\href{www.linkedin.com/in/cemeyer2}{LinkedIn}
~
contact info available upon request
~
%\emph{Please do not contact me if you work for a recruiting agency.}
\section{tongue}
native english,
professional spanish proficiency
\end{aside}

\section{objective}

\textbf{To obtain} a full time position in the software engineering field at a dynamic high tech company.

%----------------------------------------------------------------------------------------
%	WORK EXPERIENCE SECTION
%----------------------------------------------------------------------------------------
%\pagebreak
\section{industry experience}

%------------------------------------------------
\begin{entrylist}
\entry
{2020-present}
{DISCO}
{Austin, Texas}
{\emph{Senior Software Architect} \\
Rolling out an enterprise communication fabric utilizing Apache Kafka and other transit technologies to reduce development time, increase velocity, and reduce costs across the DISCO platform.}
\end{entrylist}
\begin{entrylist}
\entry
{2019-2020}
{DISCO}
{Austin, Texas}
{\emph{Software Architect} \\
Software Architect responsible for the DISCO document ingest pipeline. I focus the majority of my time around architecting and developing a serverless pipeline for high-speed parallel document ingest including file typing, conversion, OCR, and translation.}
\end{entrylist}
\begin{entrylist}
\entry
{2017-2019}
{Civitas Learning}
{Austin, Texas}
{\emph{Platform Lead, Senior Staff Software Engineer} \\
At Civitas, I was the tech lead for the data pipeline platform team. I was leading a team that was responsible for architecting and building out a next-generation streaming data pipeline in the higher education space to enable our consuming applications to be more real-time, scale efficiently, onboard new customers quicker, and reduce cloud computing costs. I was also responsible for onboarding acquisitions to the existing and next generation Civitas ETL pipelines. \\ \\
In addition to understanding the technical challenges we face, I also strived to balance out our priorities based on the business value the team could deliver. This included mentoring other engineers, technical project planning and scoping, and working with product management to move the business forward. \\ \\
Technologies used included the Confluent platform, Apache NiFi, Apache Atlas, Apache Spark, Spring boot, and the AWS ecosystem among many others.}
\end{entrylist}
\begin{entrylist}
\entry
{2014-2017}
{HomeAway}
{Austin, Texas}
{\emph{Senior Software Engineer} \\
At HomeAway, I was a senior software engineer and team lead for the internal tools and infrastructure team. I was previously on the core platform team which covers HomeAway's high-traffic REST API and enterprise messaging busses. I worked with technologies such as RabbitMQ, Elasticsearch, Hadoop, Kafka, Rails, and Docker among many others. \\
\begin{itemize}
\item Architected, developed, and rolled out a new internal application based on Ruby on Rails used by all engineers to accurately capture capitalizable time
\item Architected and developed a distributed microservice-based JSON registry used by a variety of applications across the organization
\item Led the migration to and architected new solutions on AWS for the productivity engineering organization
\item Mentored the 3 Day Startup Program sponsored by HomeAway both in Austin, TX and London, UK
\item Architected, developed, and open-sourced the \href{https://github.com/homeaway/homeaway_api_ruby}{HomeAway Ruby SDK for the HomeAway Developer API} and \href{https://github.com/homeaway/homeaway-storm}{Ruby library for communicating with Apache Storm clusters} on behalf of HomeAway
%\item Led and architected the quality effort to roll out a new caching infrastructure for our high traffic API based on Elasticsearch and Cassandra
%\item Provided continual platform support to engineers and quality assurance team members across the organization
\end{itemize}}
\end{entrylist}
\begin{entrylist}
\entry
{2012--2014}
{IBM}
{Austin, Texas}
{\emph{Staff Software Engineer} \\
Engineer on the Cluster Aware AIX team, the backbone of the clustering capabilities of the IBM enterprise grade UNIX operating system.I led the effort to maintain, refactor, and enhance the AIX cluster communications daemon. \\
\begin{itemize}
\item Led the AIX effort to certify Oracle Real Application Clusters database on PowerHA
\item Enhanced AIX clustering to scale to 32 nodes and 1024 disks per node, including parallelization of core clustering libraries
\item Enabled unicast AIX cluster heartbeating
\item Enabled dynamic network configuration change support across clustered systems
\item Co-author of a clustering algorithm patent 
\end{itemize}}
\end{entrylist}
%------------------------------------------------
\begin{entrylist}
\entry
{2010-2012}
{University of Illinois at Urbana-Champaign}
{Urbana, Illinois}
{\emph{Instructor - Department of Computer Science} \\
Taught CS528 - Object Oriented Programming and Design. \\
Graduate level course covering:
\begin{itemize}
\item Principles of object-oriented design
\item Design patterns (Gang of Four)
\item Use and design of frameworks
\item Reflection
\item Refactoring
\item Use of unit tests as specifications
\end{itemize}}
\end{entrylist}
%------------------------------------------------
\begin{entrylist}
\entry
{2010-2011}
{IBM}
{Austin, Texas}
{\emph{Power Systems Firmware Engineer Co-op} \\
Architected and developed an automated functional test suite composed of approximately 130 test cases for IBM Power Systems.}
\end{entrylist}
%------------------------------------------------
\begin{entrylist}
\entry
{2008-2009}
{IBM}
{Rochester, Minnesota}
{\emph{Power Systems Performance Co-op} \\
Worked as a member of a team to benchmark systems, analyze performance data, develop workloads, perform
maintenance on IBM Power Systems, and develop tools to aid in performance analysis. }
\end{entrylist}
%-----------------------------------------------

%----------------------------------------------------------------------------------------
%	EDUCATION SECTION
%----------------------------------------------------------------------------------------

\section{education}

\begin{entrylist}
%------------------------------------------------
\entry
{2010-2012}
{Master of Science {\normalfont computer science}}
{University of Illinois, Urbana-Champaign}
{Specialization: Object oriented architecture and design}
%------------------------------------------------
\entry
{2005-2010}
{Bachelor of Science {\normalfont computer science}}
{University of Illinois, Urbana-Champaign}
{Specialization: Software Engineering, Information Assurance}
%------------------------------------------------
\end{entrylist}
%------------------------------------------------

%----------------------------------------------------------------------------------------
%	PUBLICATIONS SECTION
%----------------------------------------------------------------------------------------

\section{publications}

\printbibsection{patent}{patents}

\printbibsection{thesis}{thesis}

\printbibsection{inproceedings}{international peer-reviewed conferences/proceedings} % Print all miscellaneous entries from the bibliography

\printbibsection{report}{technical reports} % Print all research reports from the bibliography

%----------------------------------------------------------------------------------------

\end{document}
